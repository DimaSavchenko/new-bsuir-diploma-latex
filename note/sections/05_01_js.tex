\subsection{Js}

JavaScript изначально создавался для того, чтобы сделать web-странички «живыми». Программы на этом языке называются скриптами. В браузере они подключаются напрямую к HTML и, как только загружается страничка – тут же выполняются.

Программы на JavaScript – обычный текст. Они не требуют какой-то специальной подготовки.

В этом плане JavaScript сильно отличается от другого языка, который называется Java.

JavaScript может выполняться не только в браузере, а где угодно, нужна лишь специальная программа – интерпретатор. Процесс выполнения скрипта называют «интерпретацией».

Во все основные браузеры встроен интерпретатор JavaScript, именно поэтому они могут выполнять скрипты на странице. Но, разумеется, JavaScript можно использовать не только в браузере. Это полноценный язык, программы на котором можно запускать и на сервере, и даже в стиральной машинке, если в ней установлен соответствующий интерпретатор.