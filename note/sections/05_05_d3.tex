\subsection{D3}

D3.js (или просто D3) — это JavaScript-библиотека для обработки и визуализации данных с невероятно огромными возможностями. Я, когда впервые узнал про нее, наверное, потратил не менее двух часов, просто просматривая примеры визуализации данных, созданных на D3. И конечно, когда мне самому понадобилось строить графики для небольшого внутреннего сайта на нашем предприятии, первым делом вспомнил про D3 и с мыслью, что “сейчас я всех удивлю крутейшей визуализацией”, взялся изучать исходники примеров…

… и понял, что сам абсолютно ничего не понимаю! Странная логика работы библиотеки, в примерах целая куча строк кода, чтобы создать простейший график — это был конечно же удар, главным образом по самолюбию. Ладно, утер сопли — понял, что с наскоку D3 не взять и для понимания этой библиотеки надо начинать с самых ее основ. Потому решил пойти другим путем — взять за основу для своих графиков одну из библиотек — надстроек на D3. Как выяснилось, библиотек таких довольно много — значит не один я такой, непонимающий (говорило мое поднимающееся из пепла самолюбие).

Попробовав несколько библиотек, остановился на dimple как на более или менее подходящей для моих нужд, отстроил с ее помощью все свои графики, но неудовлетворенность осталась. Некоторые вещи работали не так, как хотелось бы, другой функционал без глубокого копания в dimple не удалось реализовать, и он был отложен. Да и вообще, если нужно глубоко копать, то лучше это делать напрямую с D3, а не с дополнительной настройкой, богатый функционал которой в моем случае используется не более, чем на пять-десять процентов, а вот нужных мне настроек наоборот не хватало. И поэтому случилось то, что случилось — D3.js.