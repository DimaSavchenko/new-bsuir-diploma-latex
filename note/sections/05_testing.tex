\section{Тестирование и проверка работоспособности ПС}
\label{sec:testing}

В данном разделе проведем динамическое ручное тестирование. В таблице~\ref{table:testing:positive} приведен список тестовых случаев, относящихся к позитивному тестированию, в таблице~\ref{table:testing:negative} -- к негативному.

% Зачем: свой счетчик для нумерации тестов.
\newcounter{testnumber}
\newcommand\testnumber{\stepcounter{testnumber}\arabic{testnumber}}

% Переключаем команды нумерации для шагов тестов. В конце файла вернем всё как было.
\renewcommand{\labelenumi}{\arabic{enumi})}
\renewcommand{\labelenumii}{\asbuk{enumii})}

\begin{landscape}
	\begin{longtable}{|>{\centering}m{0.15\textwidth}
					  |p{0.8\textwidth}
					  |p{0.34\textwidth}
					  |>{\centering\arraybackslash}m{0.13\textwidth}|} 
	\caption{Тестовые случаи позитивного тестирования}
	\label{table:testing:positive}\\

	\hline
	\begin{minipage}{1\linewidth}
		\centering Модуль (экран)
	\end{minipage} & 
	\begin{minipage}{1\linewidth}
		\centering Описание тестового случая
	\end{minipage} & 
	\begin{minipage}{1\linewidth}
		\centering Ожидаемые результаты
	\end{minipage} & 
	\centering\arraybackslash Тестовый случай пройден? \endfirsthead

	\caption*{Продолжение таблицы \ref{table:testing:positive}}\\\hline
	\centering 1 & \centering 2 & \centering 3 & \centering\arraybackslash 4 \\\hline \endhead

	\hline
	\centering 1 & \centering 2 & \centering 3 & \centering\arraybackslash 4 \\

	\hline
	Аккаунт &
	\begin{minipage}[t]{1\linewidth}
		\testnumber. \textbf{Регистрация}.\newline
 		Предусловие: необходим существующий ящик электронной почты.
 		\begin{enumerate}
 			\item Нажать кнопку <<Регистрация>> на главной странице ПС.
 			\item Ввести адрес электронной почты.
 			\item Ввести пароль "12345678".
 			\item Ввести пароль из предыдущего пункта в поле подтверждения пароля.
 			\item Нажать кнопку <<Зарегистрироваться>>.
 			\item Проверить ящик электронной почты, дождаться получения электронного письма.
 			\item Перейти по ссылке из полученного письма.
 		\end{enumerate}
 	\end{minipage} &
	Отображается страница регистрации. На указанный адрес электронной почты приходит письмо со ссылкой. При переходе по ссылке появляется сообщения <<Аккаунт подтвержден>>. & Да \\
	\hline

	Аккаунт &
	\begin{minipage}[t]{1\linewidth}
		\testnumber. \textbf{Аутентификация}.\newline
		Предусловие: необходим зарегестрированный в системе аккаунт.
		\begin{enumerate}
			\item Нажать кнопку <<Вход>> на главной странице ПС.
			\item Ввести адрес электронной почты и пароль аккаунта.
			\item Нажать кнопку <<Войти>>.
		\end{enumerate}
 	\end{minipage} &
	Отображается страница аутентификации. По нажатию кнопки <<Войти>> открывается главная страница зарегистрированного пользователя. & Да \\

	Аккаунт &
	\begin{minipage}[t]{1\linewidth}
		\testnumber. \textbf{Редактирование профиля}.\newline
		Предусловие: необходимо произвести аутентификацию.
		\begin{enumerate}
			\item Нажать кнопку <<Мой профиль>>.
			\item Нажать кнопку <<Редактировать>>.
			\item Изменить значения полей имени, фамилии, отчества, описания.
			\item Обновить страницу.
			\item Нажать кнопку <<Мой профиль>>.
		\end{enumerate}
 	\end{minipage} &
	Открывается страница профиля пользователя. Открывается страница редактирования личной информации. В профиле отображается новая информация. & Да \\
	\hline

	\end{longtable}


	% Зачем: зануляем счетчик для следующей таблицы.
	\setcounter{testnumber}{0}
	
	\begin{longtable}{|>{\centering}m{0.15\textwidth}
					  |p{0.8\textwidth}
					  |p{0.34\textwidth}
					  |>{\centering\arraybackslash}m{0.13\textwidth}|} 
	\caption{Тестовые случаи негативного тестирования}
	\label{table:testing:negative}\\

	\hline
	\centering Модуль (экран) & \centering Описание тестового случая & \centeringОжидаемые результаты & \centering\arraybackslash Тестовый случай пройден? \endfirsthead

	\caption*{Продолжение таблицы \ref{table:testing:negative}}\\\hline
	\centering 1 & \centering 2 & \centering 3 & \centering\arraybackslash 4 \\\hline \endhead

	\hline
	\centering 1 & \centering 2 & \centering 3 & \centering\arraybackslash 4 \\
	\hline

	& & & \\
	\hline

	\end{longtable}
\end{landscape}

% Зачем: возвращаем нумерацию перечислений как надо по стандарту.
\renewcommand{\labelenumi}{\asbuk{enumi})}
\renewcommand{\labelenumii}{\arabic{enumii})}
