\subsection{Angular}

AngularJS представляет собой opensource JavaScript-фреймворк, использующий шаблон MVC. Собственно использование MVC является его одной из отличительных особенностей.

Для описания интерфейса используется декларативное программирование, а бизнес-логика отделена от кода интерфейса, что позволяет улучшить тестируемость и расширяемость приложений.

Другой отличительной чертой фреймворка является двустороннее связывание, позволяющее динамически изменять данные в одном месте интерфейса при изменении данных модели в другом. Таким образом, AngularJS синхронизирует модель и представление.

Кроме того, AngularJS поддерживает такие функциональности, как Ajax, управление структорой DOM, анимация, шаблоны, маршрутизация и так далее. Мощь фреймворка, наличие богатого функционала во многом повлияла на то, что он находит свое применение во все большем количестве веб-приложений, являясь на данный момент наверное одним из самых популярных javascript-фреймворков.

Официальный сайт фреймворка: http://angularjs.org/. Там вы можете найти сами исходные файлы, обучающие материалы и другие сопроводительные материалы относительно библиотеки.