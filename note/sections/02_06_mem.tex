\subsection{Memcached и Redis}
\label{sec:analysis:mem}

Memcached — программное обеспечение, реализующее сервис кэширования данных в оперативной памяти на основе хеш-таблицы.
С помощью клиентской библиотеки (для C/C++, Ruby, Perl, PHP, Python, Java, .Net и др.) позволяет кэшировать данные в оперативной памяти множества доступных серверов. Распределение реализуется путём сегментирования данных по значению хэша ключа по аналогии с сокетами хэш-таблицы. Клиентская библиотека, используя ключ данных, вычисляет хэш и использует его для выбора соответствующего сервера. Ситуация сбоя сервера трактуется как промах кэша, что позволяет повышать отказоустойчивость комплекса за счет наращивания количества memcached серверов и возможности производить их горячую замену.
В API memcached есть только базовые функции: выбор сервера, установка и разрыв соединения, добавление, удаление, обновление и получение объекта, а также Compare-and-swap. Для каждого объекта устанавливается время жизни, от 1 секунды до бесконечности. При исчерпании памяти более старые объекты автоматически удаляются. Для PHP также есть уже готовые библиотеки PECL для работы с memcached, которые дают дополнительную функциональность.

В первом приближении может показаться, что Redis мало чем отличается от Memcached. И действительно, как Redis, так и Memcached хранят данные в памяти и осуществляют доступ к ним по ключу. Оба написаны на Си и распространяются под лицензией BSD. Но в действительности, между Redis и Memcahced больше различий, чем сходства.

В первую очередь, Redis умеет сохранять данные на диск. Можно настроить Redis так, чтобы данные вообще не сохранялись, сохранялись периодически по принципу copy-on-write, или сохранялись периодически и писались в журнал (binlog). Таким образом, всегда можно добиться требуемого баланса между производительностью и надежностью.

Redis, в отличие от Memcached, позволяет хранить не только строки, но и массивы (которые могут использоваться в качестве очередей или стеков), словари, множества без повторов, большие массивы бит (bitmaps), а также множества, отсортированные по некой величине. Разумеется, можно работать с отдельными элементами списков, словарей и множеств. Как и Memcached, Redis позволяет указать время жизни данных (двумя способами — «удалить тогда-то» и «удалить через …»). По умолчанию все данные хранятся вечно.

Интересная особенность Redis заключается в том, что это — однопоточный сервер. Такое решение сильно упрощает поддержку кода, обеспечивает атомарность операций и позволяет запустить по одному процессу Redis на каждое ядро процессора. Разумеется, каждый процесс будет прослушивать свой порт. Решение нетипичное, но вполне оправданное, так как на выполнение одной операции Redis тратит очень небольшое количество времени (порядка одной стотысячной секунды).
В Redis есть репликация. Репликация с несколькими главными серверами не поддерживается. Каждый подчиненный сервер может выступать в роли главного для других. Репликация в Redis не приводит к блокировкам ни на главном сервере, ни на подчиненных. На репликах разрешена операция записи. Когда главный и подчиненный сервер восстанавливают соединение после разрыва, происходит полная синхронизация (resync).

Также Redis поддерживает транзакции (будут последовательно выполнены либо все операции, либо ни одной) и пакетную обработку команд (выполняем пачку команд, затем получаем пачку результатов). Притом ничто не мешает использовать их совместно.

Еще одна особенность Redis — поддержка механизма publish/subscribe. С его помощью приложения могут создавать каналы, подписываться на них и помещать в каналы сообщения, которые будут получены всеми подписчиками. Что-то вроде IRC-чатика.