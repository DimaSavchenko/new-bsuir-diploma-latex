\subsection{Apache и Nginx}
\label{sec:analysis:apache_nginx}

В трендах серверной разработки находятся Apache и Nginx. Это два самых распространенных веб-сервера с открытым исходным кодом в мире. Вместе они обслуживают более 50\% трафика во всем интернете. Оба решения способны работать с разнообразными рабочими нагрузками и взаимодействовать с другими приложениями для реализации полного веб-стека.

Несмотря на то, что у Apache и Nginx много схожих качеств, их нельзя рассматривать как полностью взаимозаменямые решения. Каждый из них имеет собственные преимущества и важно понимать какой веб-сервер выбрать в какой ситуации. В этой статье описано то, как каждый из этих веб-серверов ведет себя при различных условиях.

Apache HTTP Server был разработан Робертом Маккулом в 1995 году, а с 1999 года разрабатывается под управлением Apache Software Foundation — фонда развития программного обеспечения Apache. Так как HTTP сервер это первый и самый популярный проект фонда его обычно называют просто Apache.

Веб-север Apache был самым популярным веб-сервером в интернете с 1996 года. Благодаря его популярности у Apache сильная документация и интеграция со сторонним софтом.

Администраторы часто выбирают Apache из-за его гибкости, мощности и широкой распространенности. Он может быть расширен с помощью системы динамически загружаемых модулей и исполнять программы на большом количестве интерпретируемых языков программирования без использования внешнего программного обеспечения.

Nginx В 2002 году Игорь Сысоев начал работу над Nginx для того чтобы решить проблему C10K — требование к ПО работать с 10 тысячами одновременных соединений. Первый публичный релиз был выпущен в 2004 году, поставленная цель была достигнута благодаря асинхронной event-driven архитектуре.

Nginx начал набирать популярность с момента релиза благодаря своей легковесности (light-weight resource utilization) и возможности легко масштабироваться на минимальном железе. Nginx превосходен при отдаче статического контента и спроектирован так, чтобы передавать динамические запросы другому ПО предназначенному для их обработки.

Администраторы часто выбирают Nginx из-за его эффективного потребления ресурсов и отзывчивости под нагрузкой, а также из-за возможности использовать его и как веб-сервер, и как прокси.

