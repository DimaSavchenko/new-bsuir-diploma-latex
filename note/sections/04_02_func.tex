\subsection{Функциональные требования}

К разрабатываемой автоматизированной системе выдвигаются следующие функциональные требования.
\begin{enumerate}
\item[1] При открывании главной страницы сайта перед пользователем должна появиться страница с приветствием и информацией о системе. На этой же странице пользователь сможет найти ссылки на страницы создания приложения и входа в своё приложение.
Опции «Create new app» и «Login in app» представлены кнопками, при нажатии на которые пользователь перенаправляется на соответствующие страницы для совершения дальнейших действий. В верхней части страницы присутсвует хедер, на котором находится несколько полезный ссылок: Переход на главную страницу, переход на страницу документации.

\item[2] Страница создания приложения содержит поля «Email», «Password», «Confirm password», кнопку «Create».
Нажатие на кнопку «Create» в случае успеха осуществляет перенаправление пользователя на страницу с информацией о том что на электронный ящик вышлено письмо для активации аккаунта.
Для активации аккаунта необходимо перейти по ссылке в письме. После этого пользователь имеет возможность заходить в свое приложение.
Сразу после активации происходит перенаправление на страницу приложения.

\item[3] Создание приложения осуществляется путем ввода запрашиваемой информации в форму.
Все поля на форме обязательны для заполнения.
Пользователь обязан ввести почту, пароль, подтвердить пароль.
Пароль должен содержать не менее 6 знаков.
Поля «Password» и «Confirm password» должны совпадать.
Если приложение с такой почтой уже есть в системе, то должно быть выведено сообщение «Приложение с таким логином уже существует».

\item[4] Окно авторизации содержит поля «Email» и «Password», а также кнопку «Войти».
Нажатие на кнопку «Войти» в случае успеха осуществляет перенаправление пользователя на страницу приложения.

\item[5] Авторизация приложения осуществляется путем ввода запрашиваемой информации в форму авторизации.
Все поля на форме авторизации обязательны для заполнения.
Если приложения с такой почтой и паролем нет в системе, то выводится сообщение «Приложения с такой комбинацией почты и пароля не существует».
 
\item[6] На странице приложения находятся ссылки на настройки приложения. Основное поле занимает рабочая форма для составления графиков.
При нажатии на ссылку “настройки приложения” переходит переход на страницу настроек.


\item[7] Главные интерфейс представляет возможность выбора информации по разделам. 
Раздел с логами.
Раздел с docker информацией.
Раздел с n-ginx информацией.

\end{enumerate}