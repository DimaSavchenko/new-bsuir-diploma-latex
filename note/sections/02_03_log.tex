\subsection{Логирование}
\label{sec:analysis:log}
 
Естественной потребностью системного администратора или специалиста по безопасности является некий анализ того, что происходит как на конкретном компьютере конкретного пользователя, так и в локальной сети. Технически задача выполнима, ибо разработчики множества приложений, которыми мы пользуемся, заложили в свои продукты функцию логирования информации*. Информация, которую хранят логи* конкретного компьютера в сети, может сказать много тому, кто, с некоторым знанием предмета, рискнет заглянуть внутрь. Нельзя сказать, что чтение логов* является тайной дисциплиной, которая доступна только просвященным гуру, впрочем, для того, чтобы легко ориентироваться и четко сопоставлять информацию, которая встречается в логах* различных приложений, надо действительно иметь представление о том, что и как, почему и зачем пишется в логи*, а кроме того, четко представлять предметную область изучаемого ПО. Дело в том, что запись информации в логи* (вероятно, в силу некой меньшей приоритетности, чем работа самого приложения) страдает хаотичными веяниями различных производителей. Соответственно, и интерпретировать такую информацию надо с учетом специфики и, может быть, каких-то рекомендаций производителя.

Для того чтобы грамотно добывать полезную информацию из логов*, иногда достаточно простого текстового редактора и головы, но часто встречаются ситуации, когда лог* и просмотреть довольно сложно, и трактовать правильно тяжело. В этом случае полезно знать о некоторых особенностях структуры различных лог*-файлов и об информации, которая в них встречается.
