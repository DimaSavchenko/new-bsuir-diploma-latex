%Дописать введение про сервера
\sectioncentered*{ВВЕДЕНИЕ}
\addcontentsline{toc}{section}{ВВЕДЕНИЕ}

Львиная доля всех ИТ продуктов так или иначе связано с серверными технологиями. Сервера применяються для огромной области задач. Начиная от  взаимодействия играков в онлайн играх и закнчивая пересылкой электронной почты. Так же сервера это большое количество машин и кода. И важно чтобы все это стабильно работало. 

 
Серверное программное обеспечение — в информационных технологиях — программный компонент вычислительной системы, выполняющий сервисные (обслуживающие) функции по запросу клиента, предоставляя ему доступ к определённым ресурсам или услугам.
	 	 	
Понятия сервер и клиент и закреплённые за ними роли образуют программную концепцию «клиент-сервер».
Для взаимодействия с клиентом (или клиентами, если поддерживается одновременная работа с несколькими клиентами) сервер выделяет необходимые ресурсы межпроцессного взаимодействия (разделяемая память, пайп, сокет и т. п.) и ожидает запросы на открытие соединения (или, собственно, запросы на предоставляемый сервис). В зависимости от типа такого ресурса, сервер может обслуживать процессы в пределах одной компьютерной системы или процессы на других машинах через каналы передачи данных (например, COM-порт) или сетевые соединения.
Формат запросов клиента и ответов сервера определяется протоколом. Спецификации открытых протоколов описываются открытыми стандартами, например, протоколы Интернета определяются в документах RFC.
В зависимости от выполняемых задач одни серверы, при отсутствии запросов на обслуживание, могут простаивать в ожидании. Другие могут выполнять какую-то работу (например, работу по сбору информации), у таких серверов работа с клиентами может быть второстепенной задачей.



