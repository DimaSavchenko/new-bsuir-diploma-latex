\subsection{UML-Диаграммы прецедентов}

UML (англ. Unified Modeling Language – унифицированный язык моделирования) – язык графического описания для объектного моделирования в области разработки программного обеспечения, моделирования бизнес-процессов, системного проектирования и отображения организационных структур \cite{uml}. 

UML является языком широкого профиля, это – открытый стандарт, использующий графические обозначения для создания абстрактной модели системы, называемой UML-моделью. UML был создан для определения, визуализации, проектирования и документирования, в основном, программных систем. UML не является языком программирования, но на основании UML-моделей возможная генерация кода \cite{uml}. 

Использование UML позволяет также разработчикам программного обеспечения достигнуть соглашения в графических обозначениях для представления общих понятий, таких как класс, компонент, обобщение, агрегация и поведение, и больше сконцентрироваться на проектировании и архитектуре. 

Прецеденты – это технология определения функциональных требований к системе. Работа прецедентов заключается в описании типичных взаимодействий между пользователями системы и самой системой. В терминах прецедента пользователи называются актерами. Актер (actor) представляет собой некую роль, которую пользователь играет по отношению к системе. Актерами могут быть пользователь, торговый представитель пользователя, менеджер по продажам и товаровед и т.д. Актеры действуют в рамках прецедентов \cite{uml}.

% диаграмма с человечком что умеет
% диаграмма компонентов
% диаграмма авторизации




