\subsection{Базы данных}
\label{sec:analysis:db}

Можно с большой степенью достоверности утверждать, что большинство приложений, которые предназначены для выполнения хотя бы какой-нибудь полезной работы, тем или иным образом используют структурированную информацию или, другими словами, упорядоченные данные. Такими данными могут быть, например, списки заказов на тот или иной товар, списки предъявленных и оплаченных счетов или список телефонных номеров ваших знакомых. Обычное расписание движения автобусов в вашем городе - это тоже пример упорядоченных данных.

При компьютерной обработке информации упорядоченные каким либо образом данные принято хранить в базах данных - особых файлах, использование которых вместе со специальными программными средствами позволяет пользователю как просматривать необходимую информацию, так и, по мере необходимости, манипулировать ею.

Существует два основных виды баз данныйх: реляционные и нереляционные(NoSQL).

Реляционной модели данных несколько десятков лет. В реляционной модели база состоит из таблиц, которые состоят из строк и колонок. У каждой колонки есть свой тип данных (строка, число, логическое значение, дата, текст, бинарный блоб). Все строки однотипны.

Обычно каждый вид объектов хранится в отдельной таблице (например, таблица пользователей или таблица проектов). Обычно у каждого объекта есть уникальный идентификатор. Идентификатор может быть как условным, т. е. просто числом, так и вытекающим из предметной области, например номер паспорта человека или ISBN для книг. Обычно пользуются условными идентификаторами.

Объекты (таблицы) могут быть связаны друг с другом с помощью идентификатора. Например, если у нас есть таблица отделов и таблица сотрудников, то в таблице отделов есть идентификатор отдела, а в таблице сотрудников есть идентификатор сотрудника и идентификатор отдела, к которому он принадлежит. В теории реляционных баз данных этот случай называется “один ко многим” (одному отделу принадлежит много сотрудников).

Возможен также случай “многие ко многим”. Например, есть таблица проектов и таблица разработчиков. Над одним проектом могут работать много разработчиков, и один разработчик может работать над несколькими проектами. В этом случае обычно создается третья таблица — таблица связей с двумя полями: идентификатором проекта и идентификатором разработчика. Каждая связь между разработчиком и проектом выражается в виде строки в таблице связей. Если разработчик пока еще не назначен ни на один проект, то в таблице связей просто не будет ни одной записи про него.

Серверы реляционных БД обеспечивают стандартные операторы доступа к данным в таблицах, такие как SELECT, INSERT, UPDATE и DELETE. Разные серверы предоставляют также некоторые дополнительные операторы. Извлекать данные из таблиц можно по множеству различных критериев. Есть “ядро” стандарта SQL, который поддерживается практически всеми серверами, и всегда есть те или иные расширения стандарта, которые можно использовать при работе с конкретным сервером БД.

Одно из значений термина “NoSQL” — это отход от реляционной модели в пользу более специфических (или более обобщенных) моделей данных. Например, традиционно успешными NoSQL-системами являются системы хранения пар “ключ-значение”, такие как Redis или Memcache. Их модель данных предельно проста — это в сущности ассоциативный массив, где ключи имеют строковый тип, а значения могут содержать любые данные. Как и любой ассоциативный массив, такие системы поддерживают ограниченный набор операций с данными — прочитать значение по ключу, установить значение ключа, удалить ключ и связанное с ним значение. Операция “получить список ключей” может не поддерживаться в таких системах. 

Другой пример успешных NoSQL-систем — это документные хранилища. Объекты в таких хранилищах обычно являются ассоциативными массивами свободной структуры, то есть в одной и той же “таблице” могут храниться разные по сути объекты. Примеры систем такого класса — MongoDB и Cassandra. В зависимости от того, какие реально данные хранятся в конкретной базе, ее производительность может сильно варьироваться. Например, если оптимизировать такую “таблицу”, храня в ней однотипные объекты, 

Третий пример специализированных NoSQL-систем — это графовые базы данных. Они специальным образом заточены под обработку конкретной структуры данных, причем обычно для работы с большим объемом данных (потому что на небольших объемах может прекрасно справиться стандартная реляционная реализация).

Очень важным примером NoSQL-систем являются обычные файловые системы, такие как Ext4 или NTFS. Они предназначены для хранения объектов в виде иерархической структуры с содержимым свободного формата. Сами базы данных, реляционные и NoSQL, обычно используют для хранения своего содержимого именно файловые системы, и иногда взаимодействие между этими двумя подсистемами становится важным в том или ином случае.

Еще один важный случай — системы полнотекстового поиска, такие как Elastic Search или Google Search Engine. 