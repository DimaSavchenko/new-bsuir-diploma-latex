\subsection{MongoDB}

MongoDB реализует новый подход к построению баз данных, где нет таблиц, схем, запросов SQL, внешних ключей и многих других вещей, которые присущи объектно-реляционным базам данных.

Со времен динозавров было обычным делом хранить все данные в реляционных базах данных (MS SQL, MySQL, Oracle, PostgresSQL). При этом было не столь важно, а подходят ли реляционные базы данных для хранения данного типа данных или нет.

В отличие от реляционных баз данных MongoDB предлагает документо-ориентированную модель данных, благодаря чему MongoDB работает быстрее, обладает лучшей масштабируемостью, ее легче использовать.

Но, даже учитывая все недостатки традиционных баз данных и достоинства MongoDB, важно понимать, что задачи бывают разные и методы их решения бывают разные. В какой-то ситуации MongoDB действительно улучшит производительность вашего приложения, например, если надо хранить сложные по структуре данные. В другой же ситуации лучше будет использовать традиционные реляционные базы данных. Кроме того, можно использовать смешенный подход: хранить один тип данных в MongoDB, а другой тип данных - в традиционных БД.

Вся система MongoDB может представлять не только одну базу данных, находящуюся на одном физическом сервере. Функциональность MongoDB позволяет расположить несколько баз данных на нескольких физических серверах, и эти базы данных смогут легко обмениваться данными и сохранять целостность.

\subsection{SSH}

\subsection{WebSocket}