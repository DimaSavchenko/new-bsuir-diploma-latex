\sectioncentered*{Реферат}
\thispagestyle{empty}

% Добавляем три к счетчику страниц, поскольку двухстороннее задание, а также ведомость печатаются отдельно
% \FPeval{\totalpagesnumber}{round(\arabic{pagesLTS.pagenr} + 3, 0)}
% \newcommand{\totalpages}{\num{\totalpagesnumber}} 

\begin{center}
	% Пояснительная записка \totalpages~с., \num{\totfig{}}~рис., \num{\tottab{}}~табл., \num{\toteq{}}~формул, \num{\totref{}}~источника.
	\MakeUppercase{Программное средство, веб-приложение, университет, организация учебного процесса, расписание, индивидуальные задания}
\end{center}

Цель настоящего дипломного проекта состоит в разработке программной системы, предназначенной для эффективной автоматизации задач участников учебного процесса: студентов и преподавателей. 

В процессе анализа предметной области были выделены основные аспекты процесса образования в университетах, которые в настоящее время практически не охвачены автоматизацией. Было проведено их исследование и моделирование. Кроме того, рассмотрены существующие средства, разрозненно применяемые сотрудниками университетов и обучаемыми людьми (так называемые частичные аналоги). Выработаны функциональные и нефункциональные требования.

Была разработана архитектура программной системы, для каждой ее составной части было проведено разграничение реализуемых задач проектирование, уточнение используемых технологий и собственно разработка. Были выбраны наиболее современные средства разработки, широко применяемые в индустрии. 

Полученные в ходе технико-экономического обоснования результаты о прибыли для разработчика, пользователя, уровень рентабельности, а также экономический эффект доказывают целесообразность разработки проекта.
