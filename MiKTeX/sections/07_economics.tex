\newcommand{\byn}{\text{руб.}}
\newcommand{\manhour}{\text{чел./ч.}}

\newcommand{\totalloc}{\text{V}_\text{о}}
\newcommand{\normativelaboriousness}{\text{Т}_\text{н}}
\newcommand{\complexity}{\text{К}_\text{с}}
\newcommand{\stdmodules}{\text{К}_\text{т}}
\newcommand{\novelty}{\text{К}_\text{н}}
\newcommand{\totallaboriousness}{\text{Т}_\text{о}}

\newcommand{\effectivetimefund}{\text{Ф}_\text{эф}}
\newcommand{\daysinyear}{\text{Д}_\text{г}}
\newcommand{\holidays}{\text{Д}_\text{п}}
\newcommand{\weekends}{\text{Д}_\text{в}}
\newcommand{\vacationdays}{\text{Д}_\text{о}}

\newcommand{\developersnumber}{\text{Ч}_\text{р}}
\newcommand{\developmenttime}{\text{Т}_\text{р}}
\newcommand{\developertimefundsymbol}{\text{Ф}_\text{пi}}

\newcommand{\firstratetariffsymbol}{\text{Т}_\text{ч}^1}
\newcommand{\averagehourspermonthsymbol}{\text{Ф}_\text{р}}
\newcommand{\hourspershiftsymbol}{\text{Т}_\text{ч}}
\newcommand{\bonusratesymbol}{K}

\newcommand{\basewagesymbol}{\text{З}_\text{о}}
\newcommand{\additionalwageratesymbol}{\text{Н}_\text{д}}
\newcommand{\additionalwagesymbol}{\text{З}_\text{д}}

\newcommand{\ssfratesymbol}{\text{Н}_\text{сз}}
\newcommand{\ssfchargessymbol}{\text{З}_\text{сз}}
\newcommand{\insuranceratesymbol}{\text{Н}_\text{ос}}
\newcommand{\insurancechargessymbol}{\text{З}_\text{ос}}

\newcommand{\consumablesratesymbol}{\text{Н}_\text{мз}}
\newcommand{\consumableschargessymbol}{\text{М}}

\newcommand{\machinetimeratesymbol}{\text{Н}_\text{мв}}
\newcommand{\machinehourpricesymbol}{\text{Ц}_\text{м}}
\newcommand{\machinetimechargessymbol}{\text{Р}_\text{м}}

\newcommand{\businesstripratesymbol}{\text{Н}_\text{рнк}}
\newcommand{\businesstripchargessymbol}{\text{Р}_\text{нк}}

\newcommand{\otherchargesratesymbol}{\text{Н}_\text{пз}}
\newcommand{\otherchargessymbol}{\text{П}_\text{з}}

\newcommand{\overheadratesymbol}{\text{Н}_\text{нр}}
\newcommand{\overheadchargessymbol}{\text{Р}_\text{н}}

\newcommand{\totalchargessymbol}{\text{С}_\text{п}}

\newcommand{\profitabilityratesymbol}{\text{У}_\text{рп}}
\newcommand{\profitabilitysymbol}{\text{П}_\text{о}}
\newcommand{\netprofitabilitysymbol}{\text{П}_\text{ч}}

\newcommand{\netpricesymbol}{\text{Ц}_\text{п}}

\newcommand{\vatsymbol}{\text{НДС}}
\newcommand{\vatratesymbol}{\text{Н}_\text{ДС}}

\newcommand{\sellingpricesymbol}{\text{Ц}_\text{о}}

\newcommand{\deploymentchargessymbol}{\text{Р}_\text{о}}
\newcommand{\deploymentratesymbol}{\text{Н}_\text{о}}
\newcommand{\maintenancechargessymbol}{\text{Р}_\text{с}}
\newcommand{\maintenanceratesymbol}{\text{Н}_\text{с}}

\newcommand{\capitalinvestmentsymbol}{\text{К}_\text{о}}
\newcommand{\purchasecostsymbol}{\text{К}_\text{пр}}
\newcommand{\deploymentcostsymbol}{\text{К}_\text{ос}}
\newcommand{\maintenancecostsymbol}{\text{К}_\text{с}}
\newcommand{\equipmentcostsymbol}{\text{К}_\text{тс}}
\newcommand{\assetscostsymbol}{\text{К}_\text{об}}

\newcommand{\averagewagesymbol}{\text{З}_\text{см}}
\newcommand{\baselabourisnesssymbol}{\text{Т}_\text{с1}}
\newcommand{\newlabourisnesssymbol}{\text{Т}_\text{с2}}
\newcommand{\basetaskscountsymbol}{\text{A}_\text{1}}
\newcommand{\newtaskscountsymbol}{\text{A}_\text{2}}

\newcommand{\wageeconomypertasksymbol}{\text{С}_\text{зе}}
\newcommand{\wageeconomysymbol}{\text{С}_\text{з}}
\newcommand{\totalwageeconomysymbol}{\text{С}_\text{н}}

\newcommand{\basedowntimesymbol}{\text{П}_\text{1}}
\newcommand{\newdowntimesymbol}{\text{П}_\text{2}}
\newcommand{\downtimepricesymbol}{\text{С}_\text{п}}
\newcommand{\downtimechargessymbol}{\text{С}_\text{с}}
\newcommand{\planserviceworktimesymbol}{\text{Д}_\text{рг}}

\newcommand{\totaleconomysymbol}{\text{С}_\text{о}}

\newcommand{\usernetprofitsymbol}{\text{П}_\text{ч}}
\newcommand{\profittaxratesymbol}{\text{Н}_\text{п}}

\FPeval{\totalProgramSize}{28940}

\FPeval{\normativeManDays}{520}

\FPeval{\additionalComplexity}{clip(0.06+0.07)}

\FPeval{\stdModuleUsageFactor}{0.7}
\FPeval{\noveltyFactor}{0.9}

\FPeval{\daysInYear}{365}
\FPeval{\redLettersDaysInYear}{9}
\FPeval{\weekendDaysInYear}{103}
\FPeval{\vacationDaysInYear}{21}

\FPeval{\firstratetariff}{265}
\FPeval{\averagehourspermonth}{168.3}
\FPeval{\hourspershift}{8}
\FPeval{\bonusrate}{1.3}
\FPeval{\additionalwagerate}{20}
\FPeval{\ssfrate}{34}
\FPeval{\insurancerate}{0.6}
\FPeval{\consumablesrate}{5}
\FPeval{\machinetimerate}{15}
\FPeval{\machinetimereductionrate}{0.5}
\FPeval{\machinehourprice}{0.8}
\FPeval{\businesstriprate}{15}
\FPeval{\otherchargesrate}{20}
\FPeval{\overheadrate}{50}

\FPeval{\profitabilityrate}{10}
\FPeval{\vatrate}{20}

\FPeval{\deploymentrate}{10}
\FPeval{\maintenancerate}{20}

\FPeval{\averagewage}{716.5}
\FPeval{\baselabourisness}{2}
\FPeval{\newlabourisness}{0.1}
\FPeval{\usesperday}{4}

\FPeval{\basedowntime}{50}
\FPeval{\newdowntime}{10}
\FPeval{\downtimeprice}{79.8}
\FPeval{\planserviceworktime}{300}

\FPeval{\profittaxrate}{18}


\FPeval{\totalProgramSizeCorrected}{\totalProgramSize}
\newcommand{\totallocfactor}{\num{\totalProgramSizeCorrected}}

\newcommand{\additionalcomplexityfactor}{\num{\additionalComplexity}}
\FPeval{\complexityFactor}{clip(1 + \additionalComplexity)}
\newcommand{\complexityfactor}{\num{\complexityFactor}}

\newcommand{\stdmodulesfactor}{\num{\stdModuleUsageFactor}}
\newcommand{\noveltyfactor}{\num{\noveltyFactor}}

\FPeval{\adjustedManDaysExact}{clip( \normativeManDays * \complexityFactor * \stdModuleUsageFactor * \noveltyFactor )}
\FPround{\adjustedManDays}{\adjustedManDaysExact}{0}
\newcommand{\normativelaboriousnessfactor}{\num{\normativeManDays}}
\newcommand{\totallaboriousnessfactor}{\num{\adjustedManDays}}


\FPeval{\workingDaysInYear}{ clip( \daysInYear - \redLettersDaysInYear - \weekendDaysInYear - \vacationDaysInYear ) }
\newcommand{\daysinyearfactor}{\num{\daysInYear}}
\newcommand{\holidaysfactor}{\num{\redLettersDaysInYear}}
\newcommand{\weekendsfactor}{\num{\weekendDaysInYear}}
\newcommand{\vacationdaysfactor}{\num{\vacationDaysInYear}}
\newcommand{\effectivetimefundfactor}{\num{\workingDaysInYear}}

\FPeval{\requiredNumberOfProgrammers}{3}
\newcommand{\developersnumberfactor}{\num{\requiredNumberOfProgrammers}}
\FPeval{\developmentTimeYearsExact}{\adjustedManDays / (\requiredNumberOfProgrammers * \workingDaysInYear)}
\FPround{\developmentTimeYears}{\developmentTimeYearsExact}{2}
\newcommand{\developmenttimeyearsfactor}{\num{\developmentTimeYears}}
\FPeval{\developmentTimeMonthsExact}{\developmentTimeYearsExact * 12}
\FPround{\developmentTimeMonths}{\developmentTimeMonthsExact}{1}
\newcommand{\developmenttimemonthsfactor}{\num{\developmentTimeMonths}}
\FPeval{\developmentTimeDaysExact}{\developmentTimeYearsExact * \daysInYear}
\FPround{\developmentTimeDays}{\developmentTimeDaysExact}{0}
\newcommand{\developmenttimefactor}{\num{\developmentTimeDays}}

\FPeval{\developertimefund}{round(\developmentTimeDays / \requiredNumberOfProgrammers,0)}
\newcommand{\developertimefundvalue}{\num{\developertimefund}}

\FPeval{\employeeAMonthExact}{\firstratetariff * 3.98}
\FPeval{\employeeBMonthExact}{\firstratetariff * 3.48}
\FPeval{\employeeCMonthExact}{\firstratetariff * 2.65}
\FPeval{\employeeAHourExact}{\employeeAMonthExact / \averagehourspermonth}
\FPeval{\employeeBHourExact}{\employeeBMonthExact / \averagehourspermonth}
\FPeval{\employeeCHourExact}{\employeeCMonthExact / \averagehourspermonth}
\FPround{\employeeAMonth}{\employeeAMonthExact}{2}
\FPround{\employeeBMonth}{\employeeBMonthExact}{2}
\FPround{\employeeCMonth}{\employeeCMonthExact}{2}
\FPround{\employeeAHour}{\employeeAHourExact}{2}
\FPround{\employeeBHour}{\employeeBHourExact}{2}
\FPround{\employeeCHour}{\employeeCHourExact}{2}
\newcommand{\employeeamonthwage}{\num{\employeeAMonth}}
\newcommand{\employeebmonthwage}{\num{\employeeBMonth}}
\newcommand{\employeecmonthwage}{\num{\employeeCMonth}}
\newcommand{\employeeahourwage}{\num{\employeeAHour}}
\newcommand{\employeebhourwage}{\num{\employeeBHour}}
\newcommand{\employeechourwage}{\num{\employeeCHour}}

\newcommand{\firstratetariffvalue}{\num{\firstratetariff}}
\newcommand{\averagehourspermonthvalue}{\num{\averagehourspermonth}}
\newcommand{\hourspershiftvalue}{\num{\hourspershift}}
\newcommand{\bonusratevalue}{\num{\bonusrate}}

\FPeval{\basewageExact}{(\employeeAHour + \employeeBHour + \employeeCHour) * \hourspershift * \developertimefund * \bonusrate}
\FPround{\basewage}{\basewageExact}{2}
\newcommand{\basewagevalue}{\num{\basewage}}

\newcommand{\additionalwageratevalue}{\num{\additionalwagerate}\%}
\FPeval{\additionalwage}{round(\basewage * \additionalwagerate / 100, 2)}
\newcommand{\additionalwagevalue}{\num{\additionalwage}}

\FPeval{\ssfcharges}{round((\basewage + \additionalwage) * \ssfrate / 100, 2)}
\FPeval{\insurancecharges}{round((\basewage + \additionalwage) * \insurancerate / 100, 2)}
\newcommand{\ssfchargesvalue}{\num{\ssfcharges}}
\newcommand{\ssfratevalue}{\num{\ssfrate}\%}
\newcommand{\insurancechargesvalue}{\num{\insurancecharges}}
\newcommand{\insuranceratevalue}{\num{\insurancerate}\%}

\FPeval{\consumablescharges}{round(\basewage * \consumablesrate / 100, 2)}
\newcommand{\consumablesratevalue}{\num{\consumablesrate}\%}
\newcommand{\consumableschargesvalue}{\num{\consumablescharges}}

\FPeval{\machinetimecharges}{round(\machinehourprice * \totalProgramSize / 100 * \machinetimerate * \machinetimereductionrate, 2)}
\newcommand{\machinetimechargesvalue}{\num{\machinetimecharges}}
\newcommand{\machinetimeratevalue}{\num{\machinetimerate}\%}
\newcommand{\machinetimereductionratevalue}{\num{\machinetimereductionrate}}
\newcommand{\machinehourpricevalue}{\num{\machinehourprice}}

\FPeval{\businesstripcharges}{round(\basewage * \businesstriprate / 100, 2)}
\newcommand{\businesstripratevalue}{\num{\businesstriprate}\%}
\newcommand{\businesstripchargesvalue}{\num{\businesstripcharges}}

\FPeval{\othercharges}{round(\basewage * \otherchargesrate / 100, 2)}
\newcommand{\otherchargesratevalue}{\num{\otherchargesrate}\%}
\newcommand{\otherchargesvalue}{\num{\othercharges}}

\FPeval{\overheadcharges}{round(\basewage * \overheadrate / 100, 2)}
\newcommand{\overheadratevalue}{\num{\overheadrate}\%}
\newcommand{\overheadvalue}{\num{\overheadcharges}}

\FPeval{\totalcharges}{\basewage + \additionalwage + \ssfcharges + \insurancecharges + \consumablescharges + \machinetimecharges + \businesstripcharges + \othercharges + \overheadcharges}
\FPeval{\totalchargesclipped}{clip(\totalcharges)}
\newcommand{\totalchargesvalue}{\num{\totalchargesclipped}}

\FPeval{\profitability}{round(\totalcharges * \profitabilityrate / 100, 2)}
\newcommand{\profitabilityratevalue}{\num{\profitabilityrate}\%}
\newcommand{\profitabilityvalue}{\num{\profitability}}

\FPeval{\netprice}{clip(\totalcharges + \profitability)}
\newcommand{\netpricevalue}{\num{\netprice}}

\FPeval{\vat}{round(\netprice * \vatrate / 100, 2)}
\newcommand{\vatvalue}{\num{\vat}}
\newcommand{\vatratevalue}{\num{\vatrate}\%}

\FPeval{\sellingprice}{clip(\netprice + \vat)}
\newcommand{\sellingpricevalue}{\num{\sellingprice}}

\FPeval{\deploymentcharges}{round(\totalcharges * \deploymentrate / 100, 2)}
\newcommand{\deploymentratevalue}{\num{\deploymentrate}\%}
\newcommand{\deploymentchargesvalue}{\num{\deploymentcharges}}

\FPeval{\maintenancecharges}{round(\totalcharges * \maintenancerate / 100, 2)}
\newcommand{\maintenanceratevalue}{\num{\maintenancerate}\%}
\newcommand{\maintenancechargesvalue}{\num{\maintenancecharges}}

\FPeval{\netprofitability}{round(\profitability * (100 - \profittaxrate) / 100, 2)}
\newcommand{\netprofitabilityvalue}{\num{\netprofitability}}

\FPeval{\capitalinvestment}{clip(\maintenancecharges + \deploymentcharges + \sellingprice)}
\newcommand{\capitalinvestmentvalue}{\num{\capitalinvestment}}

\FPeval{\wageeconomypertask}{round(\averagewage * (\baselabourisness - \newlabourisness) / \averagehourspermonth, 2)}
\newcommand{\wageeconomypertaskvalue}{\num{\wageeconomypertask}}
\newcommand{\averagewagevalue}{\num{\averagewage}}
\newcommand{\baselabourisnessvalue}{\num{\baselabourisness}}
\newcommand{\newlabourisnessvalue}{\num{\newlabourisness}}

\FPeval{\taskscount}{clip(\usesperday * \daysInYear)}
\newcommand{\taskscountvalue}{\num{\taskscount}}
\newcommand{\usesperdayvalue}{\num{\usesperday}}

\FPeval{\wageeconomy}{clip(\wageeconomypertask * \taskscount)}
\newcommand{\wageeconomyvalue}{\num{\wageeconomy}}

\FPeval{\totalwageeconomy}{round(\wageeconomy * (100 + \bonusrate) / 100,2)}
\newcommand{\totalwageeconomyvalue}{\num{\totalwageeconomy}}

\FPeval{\downtimecharges}{round((\basedowntime - \newdowntime) * \planserviceworktime * \downtimeprice / 60,2)}
\newcommand{\downtimechargesvalue}{\num{\downtimecharges}}
\newcommand{\basedowntimevalue}{\num{\basedowntime}}
\newcommand{\newdowntimevalue}{\num{\newdowntime}}
\newcommand{\downtimepricevalue}{\num{\downtimeprice}}
\newcommand{\planserviceworktimevalue}{\num{\planserviceworktime}}

\FPeval{\totaleconomy}{clip(\totalwageeconomy + \downtimecharges)}
\newcommand{\totaleconomyvalue}{\num{\totaleconomy}}

\FPeval{\usernetprofit}{round(\totaleconomy * (1 - \profittaxrate / 100),2)}
\newcommand{\profittaxratevalue}{\num{\profittaxrate}\%}
\newcommand{\usernetprofitvalue}{\num{\usernetprofit}}

\FPeval{\usernetprofityearone}{round(\usernetprofit * 0.8696,2)}
\newcommand{\usernetprofityearonevalue}{\num{\usernetprofityearone}}
\FPeval{\usernetprofityeartwo}{round(\usernetprofit * 0.7561,2)}
\newcommand{\usernetprofityeartwovalue}{\num{\usernetprofityeartwo}}
\FPeval{\usernetprofityearthree}{round(\usernetprofit * 0.6575,2)}
\newcommand{\usernetprofityearthreevalue}{\num{\usernetprofityearthree}}

\FPeval{\excessovercostsyearzero}{clip(-\capitalinvestment)}
\newcommand{\excessovercostsyearzerovalue}{\num{\excessovercostsyearzero}}
\FPeval{\excessovercostsyearone}{clip(\usernetprofityearone)}
\newcommand{\excessovercostsyearonevalue}{\num{\excessovercostsyearone}}
\FPeval{\excessovercostsyeartwo}{clip(\usernetprofityeartwo)}
\newcommand{\excessovercostsyeartwovalue}{\num{\excessovercostsyeartwo}}
\FPeval{\excessovercostsyearthree}{clip(\usernetprofityearthree)}
\newcommand{\excessovercostsyearthreevalue}{\num{\excessovercostsyearthree}}

\FPeval{\excessovercostswithtimingyearone}{clip(\excessovercostsyearzero + \usernetprofityearone)}
\newcommand{\excessovercostswithtimingyearonevalue}{\num{\excessovercostswithtimingyearone}}
\FPeval{\excessovercostswithtimingyeartwo}{clip(\excessovercostswithtimingyearone + \usernetprofityeartwo)}
\newcommand{\excessovercostswithtimingyeartwovalue}{\num{\excessovercostswithtimingyeartwo}}
\FPeval{\excessovercostswithtimingyearthree}{clip(\excessovercostswithtimingyeartwo + \usernetprofityearthree)}
\newcommand{\excessovercostswithtimingyearthreevalue}{\num{\excessovercostswithtimingyearthree}}



\section{Технико-экономическое обоснование разработки и внедрения программного средства}
\label{sec:economics}

\subsection{Характеристика программного средства}
\label{sec:economics:description}

...

Разработки проектов программных средств связана со значительными затратами ресурсов. В связи с этим создание и реализация каждого проекта программного обеспечения нуждается в соответствующем технико-экономическом обосновании~\cite{palitsyn}, которое и описывается в данном разделе.

\subsection{Определение объема и трудоемкости ПС}
\label{sec:economics:labouriousness}

Целесообразность создания ПС требует проведения предварительной экономической оценки. Экономический эффект у разработчика ПС зависит от объема инвестиций в разработку проекта, цены на готовый продукт и количества проданных копий, и проявляется в виде роста чистой прибыли. 

Оценка стоимости создания ПС со стороны разработчика предполагает составление сметы затрат, вычисление цены и прибыли от реализации разрабатываемого программного средства. 

Исходные данные, которые будут использоваться при расчете сметы затрат, представлены в таблице~\ref{table:economics:labouriousness:initial_data}.

\begin{table}[!ht]
\caption{Исходные данные}
\label{table:economics:labouriousness:initial_data}
\centering
	\begin{tabular}{{ 
	|>{\raggedright}m{0.6\textwidth} | 
	 >{\centering}m{0.17\textwidth} | 
	 >{\centering\arraybackslash}m{0.15\textwidth}|}}

  	\hline
	{\begin{center} Наименование показателя \end{center}} & Условное обозначение &	Значение \\
  
	\hline
	Категория сложности & & 3 \\

	\hline
	Дополнительный коэффициент сложности & $\sum\limits_{i=1}^{n} \text{К}_i$ & \additionalcomplexityfactor \\

	\hline
	Степень охвата функций стандартными модулями & $\stdmodules$ & \stdmodulesfactor \\

	\hline
	Коэффициент новизны & $\novelty$ & \noveltyfactor \\

	\hline
	Количество дней в году & $\daysinyear$ & \daysinyearfactor \\

	\hline
	Количество праздничных дней в году & $\holidays$ & \holidaysfactor \\

	\hline
	Количество выходных дней в году & $\weekends$ & \weekendsfactor \\

	\hline
	Количество дней отпуска & $\vacationdays$ & \vacationdaysfactor \\

	\hline
	Количество разработчиков & $\developersnumber$ & \developersnumberfactor \\

	\hline
	Тарифная ставка первого разряда, \byn & $\firstratetariffsymbol$ & \firstratetariffvalue \\

	\hline
	Среднемесячная норма рабочего времени, ч. & $\averagehourspermonthsymbol$ & \averagehourspermonthvalue \\

	\hline
	Продолжительность рабочей смены, ч. & $\hourspershiftsymbol$ & \hourspershiftvalue \\

	\hline
	Коэффициент премирования & $\bonusratesymbol$ & \bonusratevalue \\

	\hline
	Норматив дополнительной заработной платы & $\additionalwageratesymbol$ & \additionalwageratevalue \\

	\hline
	Норматив отчислений в ФСЗН & $\ssfratesymbol $ & \ssfratevalue \\

	\hline
	Норматив отчислений по обязательному страхованию & $\insuranceratesymbol $ & \insuranceratevalue \\

	\hline
	Норматив расходов по статье <<Материалы>> & $\consumablesratesymbol $ & \consumablesratevalue \\

	\hline
	Норматив расходов по статье <<Машинное время>> & $\machinetimeratesymbol $ & \machinetimeratevalue \\

	\hline
	Понижающий коэффициент к статье <<Машинное время>> & & \machinetimereductionratevalue \\

	\hline
	Стоимость машино-часа, \byn & $\machinehourpricesymbol$ & \machinehourpricevalue \\

	\hline
	Норматив расходов по статье <<Научные командировки>> & $\businesstripratesymbol$ & \businesstripratevalue \\

	\hline
	Норматив расходов по статье <<Прочие затраты>> & $\otherchargesratesymbol$ & \otherchargesratevalue \\

	\hline
	Норматив расходов по статье <<Накладные расходы>> & $\overheadratesymbol$ & \overheadratevalue \\

	\hline
	Уровень рентабельности & $\profitabilityratesymbol$ & \profitabilityratevalue \\

	\hline
	Ставка НДС & $\vatratesymbol$ & \vatratevalue \\

	\hline
	Норматив затрат на освоение ПО & $\deploymentratesymbol$ & \deploymentratevalue \\

	\hline
	Норматив затрат на сопровождение ПО & $\maintenanceratesymbol$ & \maintenanceratevalue \\

	\hline
	\end{tabular}
\end{table}

Перед определением сметы затрат на разработку программного средства необходимо определить его объём. Однако, на стадии ТЭО нет возможности рассчитать точные объемы функций, вместо этого с помощью применения действующих нормативов рассчитываются прогнозные оценки. 

В качестве метрики измерения объема программных средств используется строка их исходного кода (LOC -- lines of code). Данная метрика широко распространена, поскольку она непосредственно связана с конечным продуктом, может применяться на всём протяжении проекта и, кроме того, может использоваться для сопоставления размеров программного обеспечения. Далее под строкой исходного кода будем понимать количество исполняемых операторов.

Расчет объема функций программного средства и общего объема приведен в таблице~\ref{table:economics:labouriousness:function_sizes}. 

\begin{table}[!ht]
\caption{Перечень и объём функций программного модуля}
\label{table:economics:labouriousness:function_sizes}
\centering
	\begin{tabular}{{ | >{\centering}m{0.12\textwidth} | 
	>{\raggedright}m{0.6\textwidth} | 
	>{\centering\arraybackslash}m{0.2\textwidth}|}}

  	\hline
	\No{} функции & 
	{\begin{center} Наименование (содержание) \end{center}} & 
	Объём функции, LoC \\
  
	\hline 
	101 & Организация ввода информации & \num{100} \\

	\hline
	102 & Контроль, предварительная обработка и ввод информации & \num{500} \\

	\hline
	109 & Организация ввода/вывода информации в интерактивном режиме & \num{190} \\

	\hline
	111 & Управление вводом/выводом & \num{2600} \\

	\hline
	204 & Обработка наборов и записей базы данных & \num{1900} \\

	\hline
	207 & Манипулирование данными & \num{8000} \\

	\hline
	208 & Организация поиска и поиск в БД & \num{7500} \\

	\hline
	304 & Обслуживание файлов & \num{500} \\

	\hline
	305 & Обработка файлов & \num{800} \\

	\hline
	309 & Формирование файла & \num{1000} \\

	\hline
	506 & Обработка ошибочных и сбойных ситуаций & \num{500} \\

	\hline
	507 & Обеспечение интерфейса между компонентами & \num{750} \\

	\hline
	601 & Отладка прикладных программ в интерактивном режиме & \num{4300} \\

	\hline
	707 & Графический вывод результатов & \num{300} \\

	\hline
	 & Общий объем & \totallocfactor \\

	\hline
	\end{tabular}
\end{table}

Исходя из определенной 3-ей категории сложности и общего объема ПС $\totalloc = \totallocfactor$, нормативная трудоемкость $\normativelaboriousness = \normativelaboriousnessfactor~\text{чел./д.}$~\cite[приложение 3]{palitsyn}. Перед определением общей трудоемкости разработки необходимо определить несколько коэффициентов.

Коэффициент сложности, который учитывает дополнительные затраты труда, связанные с обеспечением интерактивного доступа и хранения, и поиска данных в сложных структурах~\cite[приложение 4, таблица П.4.2]{palitsyn}

\begin{equation}
	\complexity = 1 + \sum_{i=1}^{n} \text{К}_i = 1 + \num{0.06} + \num{0.07} = \complexityfactor,
\end{equation}
\begin{explanation}
где & $ \text{К}_i $ & коэффициент, соответствующий степени повышения сложности за счет конкретной характеристики;\\
& $ n $ & количество учитываемых характеристик.
\end{explanation}

Коэффициент $\stdmodules$, учитывающий степень использования при разработке стандартных модулей, для разрабатываемого приложения, в котором степень охвата планируется на уровне около 50\%, примем равным \num{0.7}~\cite[приложение 4, таблица П.4.5]{palitsyn}.

Коэффициент новизны разрабатываемого программного средства $\novelty$ примем равным \noveltyfactor, так как разрабатываемом программное средство принадлежит определенному параметрическому ряду существующих программных средств~\cite[приложение 4, таблица П.4.4]{palitsyn}.

Исходя из выбранных коэффициентов, общая трудоемкость разработки $ \totallaboriousness = \normativelaboriousness \cdot \complexity \cdot \stdmodules \cdot \novelty = \normativelaboriousnessfactor \cdot \complexityfactor \cdot \stdmodulesfactor \cdot \noveltyfactor = \totallaboriousnessfactor~\text{чел./д.}$

Для расчета срока разработки проекта примем число разработчиков $\developersnumber = \developersnumberfactor$. Исходя из комментария к постановлению Министерства труда и социальной защиты Республики Беларусь от 05.10.16 №54 <<Об установлении
расчетной нормы рабочего времени на 2017 год>>~\cite{labour_calendar}, эффективный фонд времени работы одного человека составит
\begin{equation}
	\effectivetimefund = \daysinyear - \holidays - \weekends - \vacationdays = \num{\daysInYear} - \num{\redLettersDaysInYear} - \num{\weekendDaysInYear} - \num{\vacationDaysInYear} = \num{\workingDaysInYear}~\text{д.},
\end{equation}
\begin{explanation}
где & $ \text{Д}_\text{г} $ & количество дней в году;\\
& $ \text{Д}_\text{п} $ & количество праздничных дней в году;\\
& $ \text{Д}_\text{в} $ & количество выходных дней в году;\\
& $ \text{Д}_\text{о} $ & количество дней отпуска.
\end{explanation}

Тогда трудоемкость разработки проекта
\begin{equation}
	\developmenttime = \frac{\totallaboriousness}{\developersnumber \cdot \effectivetimefund} = \frac{\totallaboriousnessfactor}{\developersnumberfactor \cdot \effectivetimefundfactor} = \developmenttimeyearsfactor~\text{г.} = \developmenttimefactor~\text{д.}
\end{equation}

Исходя из того, что разработкой будет заниматься $\developersnumberfactor$ человека, можно запланировать фонд рабочего времени для каждого исполнителя
\begin{equation}
	\developertimefundsymbol = \frac{\developmenttime}{\developersnumber} = \frac{\developmenttimefactor}{\developersnumberfactor} \approx \developertimefundvalue~\text{д}.
\end{equation}

\subsection{Расчет сметы затрат}
\label{sec:economics:estimate}

Основной статьей расходов на создание ПО является заработная плата разработчиков проекта. Информация об исполнителях перечислена в таблице~\ref{table:economics:estimate:employees}. Кроме того, в таблице приведены данные об их тарифных разрядах, приведены разрядные коэффициенты, а также по формулам~\ref{eq:economics:estimate:month_wage} и~\ref{eq:economics:estimate:hour_wage} рассчитаны месячный и часовой оклады.

\begin{equation}
\label{eq:economics:estimate:month_wage}
	\text{T}_\text{м} = \text{T}_\text{м}^1 \cdot \text{T}_\text{к},
\end{equation}
\begin{equation}
\label{eq:economics:estimate:hour_wage}
	\text{T}_\text{ч} = \frac{\text{T}_\text{м}}{\text{Ф}_\text{р}},
\end{equation}
\begin{explanation}
где & $ \text{T}_\text{м} $ & месячный оклад;\\
	& $ \text{T}_\text{м}^1 $ & тарифная ставка 1-го разряда (положим ее равной \num{\firstratetariff} \byn);\\
	& $ \text{T}_\text{к} $ & тарифный коэффициент;\\
	& $ \text{T}_\text{ч} $ & часовой оклад;\\
	& $ \text{Ф}_\text{р} $ & среднемесячная норма рабочего времени (в 2017 г. составляет \num{168.3} ч.~\cite{labour_calendar}).
\end{explanation}

\begin{table}[!ht]
  \caption{Работники, занятые в проекте}
  \label{table:economics:estimate:employees}
  \begin{tabular}{| >{\raggedright}m{0.3\textwidth} 
                  | >{\centering}m{0.09\textwidth} 
                  | >{\centering}m{0.18\textwidth} 
                  | >{\centering}m{0.15\textwidth} 
                  | >{\centering\arraybackslash}m{0.15\textwidth}|}
	\hline
	{\begin{center}Исполнители\end{center}} & Разряд & Тарифный коэффициент & Месячный оклад, \byn & Часовой оклад, \byn \\

	\hline
	Руководитель проекта & 17 & \num{3.98} & \employeeamonthwage & \employeeahourwage \\

	\hline
	Ведущий инженер-программист & 15 & \num{3.48} & \employeebmonthwage & \employeebhourwage\\

	\hline
	Инженер-программист II категории & 11 & \num{2.65} & \employeecmonthwage & \employeechourwage\\
	\hline
  \end{tabular}
\end{table}

Тогда основная заработная плата исполнителей составит
\begin{equation}
\begin{aligned}
	\basewagesymbol &= \sum_{i=1}^n \text{Т}_\text{чi} \cdot \text{Т}_\text{ч} \cdot \developertimefundsymbol \cdot K = \\
	&= (\employeeahourwage + \employeebhourwage + \employeechourwage) \cdot \hourspershiftvalue \cdot \developertimefundvalue \cdot \bonusratevalue = \basewagevalue~\text{\byn},
\end{aligned}
\end{equation}
\begin{explanation}
где & $ \text{Т}_\text{чi} $ & часовая тарифная ставка i-го исполнителя, \byn;\\
	& $ \text{Т}_\text{ч} $ & количество часов работы в день;\\
	& $ \text{Ф}_\text{пi} $ & плановый фонд рабочего времени i-го исполнителя, д.;\\
	& $ K $ & коэффициент премирования (принятый равным \bonusratevalue).
\end{explanation}

Дополнительная заработная плата включает выплаты, предусмотренные законодательство о труде: оплата отпусков, льготных часов, времени  выполнения  государственных обязанностей и других выплат, не связанных с основной деятельностью исполнителей, и определяется по нормативу, установленному в организации, в процентах к основной заработной плате.
Приняв данный норматив $\additionalwageratesymbol = \additionalwageratevalue$, рассчитаем дополнительные выплаты
\begin{equation}
	\additionalwagesymbol = \frac{\basewagesymbol \cdot \additionalwageratesymbol}{100\%} = \frac{\basewagevalue \cdot \additionalwageratevalue}{100\%} = \additionalwagevalue~\byn
\end{equation}

Отчисления в фонд социальной защиты населения и в фонд обязательного страхования определяются в соответствии с действующим законодательством по нормативу в процентном отношении к фонду основной и дополнительной зарплат по следующим формулам
\begin{equation}
\begin{aligned}
	\ssfchargessymbol &= \frac{(\basewagesymbol + \additionalwagesymbol) \cdot \ssfratesymbol}{100\%},\\
	\insurancechargessymbol &= \frac{(\basewagesymbol + \additionalwagesymbol) \cdot \insuranceratesymbol}{100\%}.
\end{aligned}
\end{equation} 

В настоящее время нормы отчислений в ФСЗН $\ssfratesymbol~=~\ssfratevalue$ и в фонд обязательного страхования $\insuranceratesymbol~=~\insuranceratevalue$. Исходя из этого, размеры отчислений
\begin{equation}
\begin{aligned}
	\ssfchargessymbol &= \frac{(\basewagevalue + \additionalwagevalue) \cdot \ssfratevalue}{100\%} = \ssfchargesvalue~\byn,\\[5mm]
	\insurancechargessymbol &= \frac{(\basewagevalue + \additionalwagevalue) \cdot \insuranceratevalue}{100\%} = \insurancechargesvalue~\byn\\[5mm]
\end{aligned}
\end{equation}

Расходы по статье <<Материалы>> отражают траты на магнитные носители, бумагу, красящие материалы, необходимые для разработки ПО определяются по нормативу к фонду основной заработной платы разработчиков. Исходя из принятого норматива $\consumablesratesymbol~=~\consumablesratevalue$ определим величину расходов

\begin{equation}
	\consumableschargessymbol = \frac{\basewagesymbol \cdot \consumablesratesymbol}{100\%} = \frac{\basewagevalue \cdot \consumablesratevalue}{100\%} = \consumableschargesvalue~\byn\\[5mm]
\end{equation}

Расходы по статье <<Машинное время>> включают оплату машинного времени, необходимого для разработки и отладки ПО, которое определяется по нормативам на 100 строк исходного кода. Норматив зависит от характера решаемых задач и типа ПК; для текущего проекта примем $\machinetimeratesymbol~=~\machinetimeratevalue$~\cite[приложение 6]{palitsyn}. Примем величину стоимости машино-часа $\machinehourpricesymbol~=~\machinehourpricevalue~\byn$ Тогда, применяя понижающий коэффициент \machinetimereductionratevalue, получим величину расходов
\begin{equation}
	\machinetimechargessymbol = \machinehourpricesymbol \cdot \frac{\totalloc}{100} \cdot \machinetimeratesymbol = \machinehourpricevalue \cdot \frac{\totallocfactor}{100} \cdot \machinetimeratevalue \cdot \machinetimereductionratevalue = \machinetimechargesvalue~\byn
\end{equation}

Расходы по статье <<Научные командировки>> определяются по нормативу в процентах к основной заработной плате. Принимая норматив равным $\businesstripratesymbol~=~\businesstripratevalue$ получим величину расходов
\begin{equation}
	\businesstripchargessymbol = \frac{\basewagesymbol \cdot \businesstripratesymbol}{100\%} = \frac{\basewagevalue \cdot \businesstripratevalue}{100\%} = \businesstripchargesvalue~\byn
\end{equation}

Расходы по статье <<Прочие затраты>> включают затраты на приобретение и подготовку специальной научно-технической информации и специальной литературы. Определяются по нормативу в процентах к основной заработной плате. Принимая норматив равным $\otherchargesratesymbol~=~\otherchargesratevalue$ получим величину расходов
\begin{equation}
	\otherchargessymbol = \frac{\basewagesymbol \cdot \otherchargesratesymbol}{100\%} = \frac{\basewagevalue \cdot \otherchargesratevalue}{100\%} = \otherchargesvalue~\byn
\end{equation}

Затраты по статье <<Накладные расходы>>, связанные с необходимостью  содержания  аппарата  управления,  вспомогательных хозяйств и опытных (экспериментальных) производств, а также с расходами на общехозяйственные нужды, относятся к конкретному ПО по нормативу в процентном отношении к основной заработной плате
исполнителей. Принимая норматив равным $\overheadratesymbol~=~\overheadratevalue$ получим величину расходов
\begin{equation}
	\overheadchargessymbol = \frac{\basewagesymbol \cdot \overheadratesymbol}{100\%} = \frac{\basewagevalue \cdot \overheadratevalue}{100\%} = \overheadvalue~\byn
\end{equation}

Общая сумма расходов по смете определяется как сумма вышерассчитанных показателей

\begin{equation}
\begin{aligned}
	\totalchargessymbol = \basewagesymbol + \additionalwagesymbol + \ssfchargessymbol &+ \insurancechargessymbol + \consumableschargessymbol + \machinetimechargessymbol + \businesstripchargessymbol + \otherchargessymbol + \overheadchargessymbol =\\
	&= \totalchargesvalue~\byn
\end{aligned}
\end{equation}

Рентабельность определяется из результатов анализа рыночных условий и переговоров с потребителями ПО. Исходя из принятого уровня рентабельности $\profitabilityratesymbol~=~\profitabilityratevalue$, прибыль от реализации ПО составит
\begin{equation}
	\profitabilitysymbol = \frac{\totalchargessymbol \cdot \profitabilityratesymbol}{100\%} = \frac{\totalchargesvalue \cdot \profitabilityratevalue}{100\%} = \profitabilityvalue~\byn
\end{equation}

На основании расчета прибыли и уровня себестоимости рассчитаем прогнозируемую цену программного средства без учета налогов
\begin{equation}
	\netpricesymbol = \totalchargessymbol + \profitabilitysymbol = \totalchargesvalue + \profitabilityvalue = \netpricevalue~\byn
\end{equation}

Далее рассчитаем налог на добавленную стоимость
\begin{equation}
	\vatsymbol = \frac{\netpricesymbol \cdot \vatratesymbol}{100\%} = \frac{\netpricevalue \cdot \vatratevalue}{100\%} = \vatvalue~\byn
\end{equation}

НДС включается в прогнозируемую отпускную цену
\begin{equation}
	\sellingpricesymbol = \netpricesymbol + \vatsymbol = \netpricevalue + \vatvalue = \sellingpricevalue~\byn
\end{equation}

Организация-разработчик участвует в освоении и внедрении ПО и несет соответствующие затраты, которые определяются по нормативу $\deploymentratesymbol~=~\deploymentratevalue$ от себестоимости ПО в расчете на три месяца
\begin{equation}
	\deploymentchargessymbol = \frac{\totalchargessymbol \cdot \deploymentratesymbol}{100\%} = \frac{\totalchargesvalue \cdot \deploymentratevalue}{100\%} = \deploymentchargesvalue~\byn
\end{equation}

Кроме того, организация-разработчик осуществляет сопровождение ПО, которое также оплачивается заказчиком. Расчет осуществляется в соответствии с нормативом $\maintenanceratesymbol~=~\maintenanceratevalue$ от себестоимости ПО
\begin{equation}
	\maintenancechargessymbol = \frac{\totalchargessymbol \cdot \maintenanceratesymbol}{100\%} = \frac{\totalchargesvalue \cdot \maintenanceratevalue}{100\%} = \maintenancechargesvalue~\byn
\end{equation}

Экономическим эффектом разработчика будет являться сумма прибыли с вычетом налога на прибыль
\begin{equation}
	\netprofitabilitysymbol = \profitabilitysymbol - \frac{\profitabilitysymbol \cdot \profittaxratesymbol}{100\%} = \profitabilityvalue - \frac{\profitabilityvalue \cdot \profittaxratevalue}{100\%} = \netprofitabilityvalue~\byn
\end{equation}

\subsection{Оценка экономической эффективности применения ПС у пользователя}
\label{sec:economics:effect}

В результате применения нового ПО пользователь может понести значительные капитальные затраты на приобретение и освоение ПО, доукомплектованием ЭВМ новыми техническими средствами и пополнение оборотных средств. Однако, если приобретенное ПО будет в достаточной степени эффективнее базового, то дополнительные капитальные затраты быстро окупятся.

Для определения экономического эффекта от использования нового ПО у потребителя необходимо сравнить расходы по всем основным статьям сметы затрат на эксплуатацию нового ПО с расходами по соответствующим статьям базового варианта. При этом за базовый вариант примем ручной вариант. Исходные данные для расчета приведены в таблице~\ref{table:economics:effect:initial_data}.

\begin{table}[!ht]
\caption{Исходные данные}
\label{table:economics:effect:initial_data}
\centering
	\begin{tabular}{{ 
	|>{\raggedright}m{0.41\textwidth} | 
	 >{\centering}m{0.17\textwidth} | 
	 >{\centering}m{0.14\textwidth} | 
	 >{\centering\arraybackslash}m{0.18\textwidth}|}}

  	\hline
	{\begin{center} Наименование показателя \end{center}} & Условное обозначение & Значение в базовом варианте & Значение в новом варианте \\

	\hline
	Затраты пользователя на приобретение ПО & $\purchasecostsymbol$ & & \sellingpricevalue~\byn\\

	\hline
	Затраты пользователя на освоение & $\deploymentcostsymbol$ & & \deploymentchargesvalue~\byn\\

	\hline
	Затраты пользователя на сопровождение & $\maintenancecostsymbol$ & & \maintenancechargesvalue~\byn\\

	\hline
	Трудоемкость на задачу,~\manhour & $\baselabourisnesssymbol,\newlabourisnesssymbol$ & \baselabourisnessvalue & \newlabourisnessvalue\\

	\hline
	Средняя зарплата\tablefootnote{На февраль 2017 г.~\cite{belstat_average_wage}},~\byn & $\averagewagesymbol$ & \averagewagevalue & \averagewagevalue\\

	\hline
	Количество выполняемых задач & $\basetaskscountsymbol,\newtaskscountsymbol$ & \taskscountvalue & \taskscountvalue\\

	\hline
	Время простоя сервиса, мин. в день & $\basedowntimesymbol,\newdowntimesymbol$ & \basedowntimevalue & \newdowntimevalue\\

	\hline
	Стоимость одного часа простоя, \byn & $\downtimepricesymbol$ & \downtimepricevalue & \downtimepricevalue\\

	\hline
	\end{tabular}
\end{table}

Общие капитальные вложения заказчика (потребителя) вычисляются следующим образом
\begin{equation}
\begin{aligned}
	\capitalinvestmentsymbol = \purchasecostsymbol + \deploymentcostsymbol + \maintenancecostsymbol &= \sellingpricevalue + \deploymentchargesvalue + \maintenancechargesvalue = \\
	&= \capitalinvestmentvalue~\byn
\end{aligned}
\end{equation}
\begin{explanation}
где &$ \purchasecostsymbol $& затраты пользователя на приобретение ПО по отпускной цене;\\
	&$ \deploymentcostsymbol $& затраты пользователя на освоение;\\
	&$ \maintenancecostsymbol $& затраты пользователя на оплату услуг по сопровождению.
\end{explanation}

В качестве типичного примера использования разрабатываемого ПС предполагается сценарий ее использования ежедневно $\usesperday$ раза в день, при этом предполагается снижение трудоемкости с $\baselabourisnesssymbol~=~\baselabourisnessvalue~\manhour$ до\\ $\newlabourisnesssymbol~=~\newlabourisnessvalue~\manhour$. Приняв среднемесячную заработную плату работника $\averagewagesymbol~=~\averagewagevalue~\byn$, рассчитаем экономию затрат на заработную плату в расчете на одну задачу $\wageeconomypertasksymbol$ и за год $\wageeconomysymbol$ 
\begin{equation}
	\wageeconomypertasksymbol = \frac{\averagewagesymbol \cdot (\baselabourisnesssymbol - \newlabourisnesssymbol)}{\averagehourspermonthsymbol} = \frac{\averagewagevalue \cdot (\baselabourisnessvalue - \newlabourisnessvalue)}{\averagehourspermonthvalue} = \wageeconomypertaskvalue~\byn,
\end{equation}
\begin{equation}
	\wageeconomysymbol = \wageeconomypertasksymbol \cdot \newtaskscountsymbol = \wageeconomypertaskvalue \cdot \taskscountvalue = \wageeconomyvalue~\byn,
\end{equation}
\begin{explanation}
где &$\averagehourspermonthsymbol$& среднемесячная норма рабочего времени, ч.
\end{explanation}

Экономия с учетом начислений на зарплату
\begin{equation}
	\totalwageeconomysymbol = \wageeconomysymbol \cdot \frac{100\% + \bonusratesymbol}{100\%} = \wageeconomyvalue \cdot \frac{100\% + \bonusratevalue}{100\%} = \totalwageeconomyvalue~\byn,
\end{equation}
\begin{explanation}
где &$\bonusratesymbol$& норматив начислений на зарплату, \byn
\end{explanation}

Экономия за счет сокращения простоев сервиса
\begin{equation}
	\downtimechargessymbol = \frac{(\basedowntimesymbol - \newdowntimesymbol) \cdot \planserviceworktimesymbol \cdot \downtimepricesymbol}{60} = \frac{(\basedowntimevalue - \newdowntimevalue) \cdot \planserviceworktimevalue \cdot \downtimepricevalue}{60} = \downtimechargesvalue,~\byn,
\end{equation}
\begin{explanation}
где &$\planserviceworktimesymbol$& плановый фонд работы сервиса, д.
\end{explanation}

Тогда общая годовая экономия текущих затрат, связанных с использованием нового ПО
\begin{equation}
	\totaleconomysymbol = \totalwageeconomysymbol + \downtimechargessymbol = \totalwageeconomyvalue + \downtimechargesvalue = \totaleconomyvalue~\byn
\end{equation}

Внедрение нового ПО позволит пользователю сэкономить на текущих затратах, то есть получить на эту сумму дополнительную прибыль. Для пользователя в качестве экономического эффекта выступает лишь чистая прибыль. Принимая размер ставки налога на прибыль $\profittaxratesymbol~=~\profittaxratevalue$ получим
\begin{equation}
	\Delta\usernetprofitsymbol = \totaleconomysymbol - \frac{\totaleconomysymbol \cdot \profittaxratesymbol}{100\%} = \totaleconomyvalue - \frac{\totaleconomyvalue \cdot \profittaxratevalue}{100\%} = \usernetprofitvalue~\byn
\end{equation}

В процессе использования нового ПО чистая прибыль в конечном итоге возмещает капитальные затраты. Однако полученные при этом суммы прибыли и затрат по годам приводят к единому времени -- расчетному году (за расчетный год принят 2017-й год) путем умножения результатов и затрат за каждый год на коэффициент дисконтирования $\alpha$. При расчете используются следующие коэффициенты: $2017~\text{г.} - \num{1}, 2018~\text{г.} - \num{0.8696}, 2019~\text{г.} - \num{0.7561}, 2020~\text{г.} - \num{0.6575}$. Все рассчитанные данные экономического эффекта сводятся в таблицу~\ref{table:economics:effect:final_data}.

\begin{table}[!ht]
\caption{Расчет экономического эффекта от использования нового ПО}
\label{table:economics:effect:final_data}
\centering
	\begin{tabular}{{ 
	|>{\raggedright}m{0.32\textwidth} | 
	 >{\centering}m{0.1375\textwidth} | 
	 >{\centering}m{0.1375\textwidth} | 
	 >{\centering}m{0.1375\textwidth} | 
	 >{\centering\arraybackslash}m{0.1375\textwidth}|}}

  	\hline
	{\begin{center} Наименование показателя \end{center}} & 2017 г. & 2018 г. & 2019 г. & 2020 г. \\

	\hline
	\emph{Результаты} & & & & \\

	\hline
	Коэффициент приведения & \num{1} & \num{0.8696} & \num{0.7561} & \num{0.6575} \\

	\hline
	Прирост прибыли за счет экономии затрат ($\usernetprofitsymbol$),~\byn & & \usernetprofitvalue & \usernetprofitvalue & \usernetprofitvalue \\

	\hline
	То же с учетом фактора времени,~\byn  & & \usernetprofityearonevalue & \usernetprofityeartwovalue & \usernetprofityearthreevalue \\

	\hline
	\emph{Затраты} & & & & \\

	\hline
	Приобретение ПО ($\purchasecostsymbol$),~\byn & \sellingpricevalue & & & \\

	\hline
	Освоение ПО ($\deploymentcostsymbol$),~\byn & \deploymentchargesvalue & & & \\

	\hline
	Сопровождение ПО ($\maintenancecostsymbol$),~\byn & \maintenancechargesvalue & & & \\

	\hline
	Всего затрат ($\capitalinvestmentsymbol$),~\byn & \capitalinvestmentvalue & & & \\

	\hline
	То же с учетом фактора времени,~\byn & \capitalinvestmentvalue & & & \\

	\hline
	\emph{Экономический эффект} & & & & \\

	\hline
	Превышение результата над затратами,~\byn & \excessovercostsyearzerovalue & \excessovercostsyearonevalue & \excessovercostsyeartwovalue & \excessovercostsyearthreevalue \\

	\hline
	То же с нарастающим итогом,~\byn & \excessovercostsyearzerovalue & \excessovercostswithtimingyearonevalue & \excessovercostswithtimingyeartwovalue & \excessovercostswithtimingyearthreevalue \\

	\hline
	\end{tabular}
\end{table}

В результате технико-экономического обоснования применения программного средства были получены следующие значения показателей эффективности:

\begin{itemize}
	\item чистая прибыль разработчика составит $\netprofitabilitysymbol~=~\netprofitabilityvalue~\byn$;
	\item затраты заказчика окупятся уже на четвертом году использования;
	\item экономическая эффективность для заказчика, выраженная в виде чистого дисконтированного дохода, составит $\excessovercostswithtimingyearthreevalue~\byn$ за четыре года использования данного ПС; более высокий прирост прибыли заказчик получит по истечению данного срока.
\end{itemize}

Полученные результаты свидетельствуют об эффективности разработки и внедрения проектируемого программного средства.
