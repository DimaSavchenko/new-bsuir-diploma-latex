\subsection{Аналитический обзор литературных источников}
\label{sec:analysis:literature}

Далее приводится анализ сведений, которые влияют на формулирование требований, выбор архитектуры и дальнейшее проектирование и разработку программного средства.

...

Для проектируемого программного средства актуальны следующие характеристики:
\begin{itemize}
  \item нет необходимости в организации ресурсоемких вычислений;
  \item желательна возможность использования мгновенных уведомлений и оповещений;
  \item желательна доступность приложения на различных компьютерах пользователя.
\end{itemize}

...

При проектировании архитектуры программной системы почти никогда не ограничиваются единственным архитектурным стилем, поскольку они могут предлагать решение каких-либо проблем в различных областях. В таблице~\ref{table:analysis:architectures:categorization} приведен вариант категоризации архитектурных стилей.

\begin{table}[ht]
\caption{Категоризация архитектурных стилей}
\label{table:analysis:architectures:categorization}
\centering
  \begin{tabular}{|>{\raggedright}m{0.27\textwidth} 
                  |>{\raggedright\arraybackslash}m{0.68\textwidth}|}
  \hline Категория & Архитектурный стиль \\
  \hline Связь & SOA\tablefootnote{Service-oriented architecture -- архитектура, ориентированная на сервисы}, Шина сообщений \\
  \hline Развертывание & Клиент-серверный, трехуровневый, N-уровневый \\
  \hline Предметная область & DDD\tablefootnote{Domain-driven design -- проблемно-ориентированное проектирование} \\
  \hline Структура & Компонентный, объектно-ориентированный, многоуровневый\\
  \hline
  \end{tabular}
\end{table}

...
